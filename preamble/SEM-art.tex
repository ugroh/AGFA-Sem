%% %%%%%%%%%%%%%%%%%%%%%%%%%%%%%%%%%%%%%%%%%%%%%%%%%%%%%%%%%%
%% AGFA-Seminar.tex
%% Layout, Font, Pakete etc.
%% Stand 2022/10/10
%% %%%%%%%%%%%%%%%%%%%%%%%%%%%%%%%%%%%%%%%%%%%%%%%%%%%%%%%%%%

\usepackage{onlyamsmath}			% Fehler bei Eingabe Mathematik
\usepackage[l2tabu,orthodox]{nag} 	% Fehler im LaTeX-Code

%% --------------------------
%% Falls kein UTF-8 eingestellt ist
%% Kann entfallen bei UTF8
%% --------------------------
\usepackage{selinput} 		% Automatische Wahl der "encoding"
	\SelectInputMappings{	% texdoc selinput	
		,adieresis={ä}		% 
       	,germandbls={ß}		%
            }
\usepackage[T1]{fontenc}   % texdoc fontenc

%% --------------------------
%% Wir schreiben deutschen Text und nutzen die deutschen Trennungsregeln
%% --------------------------
\usepackage[ngerman]{babel}
\usepackage[babel,german=guillemets]{csquotes} 		% \enquote{Text}
\babelprovide[hyphenrules=ngerman-x-latest]{ngerman}

%% --------------------------
%% Libertinus
%% --------------------------
\usepackage{libertinus}			

%% --------------------------	
%% -- Alternative lmodern 
%% --------------------------	
%\usepackage{lmodern}
%\DeclareMathVersion{sans}
%	\SetSymbolFont{letters}{sans}{OML}{cmbr}{m}{it} % Math letters from Latin Modern Sans
%	\SetSymbolFont{operators}{sans}{OT1}{lmss}{m}{n} % Math operators
%	\SetSymbolFont{symbols}{sans}{OMS}{lmsy}{m}{n} % Math symbols
%	\SetMathAlphabet{\mathrm}{sans}{OT1}{lmr}{m}{n} % Large symbols
%	\SetMathAlphabet{\mathsf}{sans}{OT1}{lmss}{m}{n}
%	\SetMathAlphabet{\mathit}{sans}{OT1}{lmr}{m}{it}

%% --------------------------
%% Layout siehe KOMA-Script Buch für Details
%% --------------------------
\addtokomafont{title}{\let\huge\Large} 
\addtokomafont{author}{\large} 
\addtokomafont{date}{\normalsize}
% 
\renewcommand*{\thesubsection}{\arabic{subsection}.\kern1pt}
\RedeclareSectionCommand[font = \itshape]{subsection}
\setcounter{secnumdepth}{\subsubsectionnumdepth} 
%
\renewcommand*{\thesubsubsection}{\arabic{subsubsection}.\kern1pt} 
%
\RedeclareSectionCommand[%
  	,beforeskip		= .5\baselineskip 	% Etwas Abstand
	,afterskip		= -0.1em			% Spitzmarke, d.h. keine Absatz
	,font			= \normalfont	%	
	]{subsubsection} 

%% --------------------------
%% -- Kopf/Fußzeile
%% --------------------------

%% --------------------------
%% Sinnvolle Pakete
%% --------------------------

%% Bessere Aufzählungen 
\usepackage[shortlabels,inline]{enumitem} 	% texdoc enumitem

%% Flattersatz
\usepackage[%
	,newcommands
	,footnotes
	,raggedrightboxes
		]{ragged2e}	% Für Flattersatz

%% --------------------------
%% Mathematik
%% --------------------------
\usepackage{amsmath,amsthm,amssymb}

%% -- Mathematische Umgebungen	texdoc amsthm
\theoremstyle{plain}
%
\newtheorem{theorem}{Theorem}
\newtheorem{thm}[theorem]{Theorem}
\newtheorem{proposition}[theorem]{Satz}
\newtheorem{prop}[theorem]{Satz}
\newtheorem{satz}[theorem]{Satz}
\newtheorem{corollary}[theorem]{Korollar}
\newtheorem{cor}[theorem]{Korollar}
\newtheorem{korollar}[theorem]{Korollar}
\newtheorem{lemma}[theorem]{Lemma}
\newtheorem{lem}[theorem]{Lemma}
%
\theoremstyle{definition}
%
\newtheorem{definition}[theorem]{Definition}
\newtheorem{defn}[theorem]{Definition}
\newtheorem{example}[theorem]{Beispiel}
\newtheorem{beispiel}[theorem]{Beispiel}
%
\theoremstyle{remark}
%
\newtheorem*{note}{Anmerkung}
\newtheorem*{rem}{Anmerkung}
\newtheorem*{remark}{Anmerkung}

% Hier unnötig, da bereits über amsthm definiert
%\newtheorem{proof}{Beweis}
%\theoremsymbol{\ensuremath{\square}}















