%% %%%%%%%%%%%%%%%%%%%%%%%%%%%%%%%%%%%%%%%%%%%%%%%%%%%%%%%%%%
%% Einige sinnvolle Definitionen
%% und nützliche Pakete
%% Eigene Ergäzungen in My-Def.tex
%% Stand: 2022/10/10
%% %%%%%%%%%%%%%%%%%%%%%%%%%%%%%%%%%%%%%%%%%%%%%%%%%%%%%%%%%%
%% -- Mathematische Abkürzungen
\usepackage{mathtools}
%% -- Norm, Absolutbetrag, duales Paar siehe mathtools Abschnitt 3.6
\DeclarePairedDelimiterX{\norm}[1]{\lVert}{\rVert}{\ifblank{#1}{\:\cdot\:}{#1}}
\DeclarePairedDelimiterX{\abs}[1]{\lvert}{\rvert}{\ifblank{#1}{\:\cdot\:}{#1}}

%% -- Intervalle 
\newcommand{\interval}[1]{\left[ #1 \right]}	
\newcommand{\ointerval}[1]{\left] #1 \right[}	
\newcommand{\rointerval}[1]{\left[ #1 \right[}
\newcommand{\lointerval}[1]{\left] #1 \right]} 

%% -- Zahlen

\newcommand{\N}{\mathbb{N}}		% Natürliche Zahlen
\newcommand{\Z}{\mathbb{Z}}		% Ganze Zahlen
\newcommand{\Q}{\mathbb{Q}}		% Rationale Zahlen
\newcommand{\R}{\mathbb{R}}		% Reelle Zahlen
\newcommand{\C}{\mathbb{C}}		% Komplexe Zahlen
\newcommand{\K}{\mathbb{K}}		% Koerperzeichen

%% -- Ableitungen
%% -- auch als Beispiel für mathematische Ausdrücke
 
\newcommand*{\ds}{\mathop{}\!\mathrm{d}{s}}       % \ds = ds, 
\newcommand*{\dt}{\mathop{}\!\mathrm{d}{t}}       % \dt = dt, 
\newcommand*{\dx}{\mathop{}\!\mathrm{d}{x}}       % \dx = dx,
\newcommand*{\diff}[1]{\mathop{}\!\mathrm{d}{#1}}	% \diff{\mu} = d\mu, 

%% var-Symbole anstelle der Originalen, besser zu unterscheiden
%% --

\let\ORGvarepsilon=\varepsilon
\let\varepsilon=\epsilon
\let\epsilon=\ORGvarepsilon
%
\let\ORGvarrho=\varrho
\let\varrho=\rho
\let\rho=\ORGvarrho
%
\let\ORGvartheta=\vartheta
\let\vartheta=\theta
\let\theta=\ORGvartheta
%
\let\ORGvarphi=\varphi
\let\varphi=\phi
\let\phi=\ORGvarphi
%
\let\ORGvarleq=\leqslant
\let\leqslant=\leq
\let\leq=\ORGvarleq
%
\let\ORGvargeq=\geqslant
\let\geqslant=\geq
\let\geq=\ORGvargeq

%% -- Richtigen Abkürzungen; bitte auch in xspace-manual reinsehen
%% -- Bei Verwendung von \usepackage{hyperref} nach diesem Paket laden
%% --

\usepackage{xspace}
\newcommand{\zB}{z.\,B.\xspace}
\newcommand{\og}{o.\,g.\xspace}
\renewcommand{\dh}{d.\,h.\xspace}
\newcommand{\ua}{\mbox{u.\,a.}\xspace}
\newcommand{\etc}{etc.\xspace}
\newcommand{\bzw}{bzw.\xspace}
\newcommand{\vs}{vs.\xspace}

