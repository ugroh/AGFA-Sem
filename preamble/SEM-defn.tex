%% %%%%%%%%%%%%%%%%%%%%%%%%%%%%%%%%%%%%%%%%%%%%%%%%%%%%%%%%%%
%% Einige sinnvolle Definitionen
%% und nützliche Pakete
%% Eigene Ergäzungen in My-Def.tex
%% Stand: 2023/01/16
%% %%%%%%%%%%%%%%%%%%%%%%%%%%%%%%%%%%%%%%%%%%%%%%%%%%%%%%%%%%
%% -- Mathematische Abkürzungen
\usepackage{mathtools}
%% -- Norm, Absolutbetrag, siehe mathtools Abschnitt 3.6
%% -- angepasst an größere Symbole via der Stern-Variante \norm*{}
%% --
\DeclarePairedDelimiterX{\norm}[1]{\lVert}{\rVert}{\ifblank{#1}{\:\cdot\:}{#1}}
\DeclarePairedDelimiterX{\abs}[1]{\lvert}{\rvert}{\ifblank{#1}{\:\cdot\:}{#1}}

%% -- Intervalle 
%% -- da muss man sich keine Gedanken über die Reihenfolge der 
%% -- Klammern machen; sonst halt \left]a,b\right] für ]a,b] erforderlich
%% --
\newcommand{\interval}[1]{\left[ #1 \right]}	
\newcommand{\ointerval}[1]{\left] #1 \right[}	
\newcommand{\rointerval}[1]{\left[ #1 \right[}
\newcommand{\lointerval}[1]{\left] #1 \right]} 

%% -- Zahlen
%% --
\newcommand{\N}{\mathbb{N}}		% Natürliche Zahlen
\newcommand{\Z}{\mathbb{Z}}		% Ganze Zahlen
\newcommand{\Q}{\mathbb{Q}}		% Rationale Zahlen
\newcommand{\R}{\mathbb{R}}		% Reelle Zahlen
\newcommand{\C}{\mathbb{C}}		% Komplexe Zahlen
\newcommand{\K}{\mathbb{K}}		% Koerperzeichen

%% -- Ableitungen
%% -- auch als Beispiel für mathematische Ausdrücke
 
\newcommand*{\ds}{\mathop{}\!\mathrm{d}{s}}       % \ds = ds, 
\newcommand*{\dt}{\mathop{}\!\mathrm{d}{t}}       % \dt = dt, 
\newcommand*{\dx}{\mathop{}\!\mathrm{d}{x}}       % \dx = dx,
\newcommand*{\diff}[1]{\mathop{}\!\mathrm{d}{#1}}	% \diff{\mu} = d\mu, 
\renewcommand*{\d}[1]{\mathop{}\!\mathrm{d}{#1}}	% \d{\mu} = d\mu, 

%% -- var-Symbole anstelle der Originale, besser zu unterscheiden
%% -- Bitte beachten: \varphi gibt nun etwas \phi
%% --

\let\ORGvarepsilon=\varepsilon
\let\varepsilon=\epsilon
\let\epsilon=\ORGvarepsilon
%
\let\ORGvarrho=\varrho
\let\varrho=\rho
\let\rho=\ORGvarrho
%
\let\ORGvartheta=\vartheta
\let\vartheta=\theta
\let\theta=\ORGvartheta
%
\let\ORGvarphi=\varphi
\let\varphi=\phi
\let\phi=\ORGvarphi
%
\let\ORGvarleq=\leqslant
\let\leqslant=\leq
\let\leq=\ORGvarleq
%
\let\ORGvargeq=\geqslant
\let\geqslant=\geq
\let\geq=\ORGvargeq

%% -- Richtigen Abkürzungen; bitte auch in das xspace-manual reinsehen
%% -- Bei Verwendung von \usepackage{hyperref} nach diesem Paket laden
%% -- \mbox{} damit nicht getrennt wird
%% --
\usepackage{xspace}
%%
\newcommand{\zB}{\mbox{z.\,B.}\xspace}
\newcommand{\iA}{\mbox{i.\,A.}\xspace}
\newcommand{\iAllg}{\mbox{i.\,Allg.}\xspace}
\renewcommand{\dh}{\mbox{d.\,h.}\xspace}
\newcommand{\oAe}{\mbox{o.\,Ä.}\xspace}
\newcommand{\uAe}{\mbox{u.\,Ä.}\xspace}
\newcommand{\og}{\mbox{o.\,g.}\xspace}
\newcommand{\ua}{\mbox{u.\,a.}\xspace} 
%%
\newcommand{\inkl}{inkl.\xspace} 
\newcommand{\sog}{sog.\xspace} 
\newcommand{\bzgl}{bzgl.\xspace} 
\newcommand{\fue}{\mbox{f.\,ü.}\xspace}
\newcommand{\vs}{vs.\xspace} 
\newcommand{\ca}{ca.\xspace}
\newcommand{\bzw}{bzw.\xspace}
\newcommand{\etc}{etc.\xspace}
\newcommand{\usw}{usw.\xspace}
\newcommand{\ggf}{ggf.\xspace}
\newcommand{\evtl}{evtl.\xspace}

