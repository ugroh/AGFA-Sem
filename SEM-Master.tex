% !TEX TS-program = pdflatexmk
%% %%%%%%%%%%%%%%%%%%%%%%%%%%%%%%%%%%%%%%%%%%%%%%%%%%%%%%%%%%%%%%%% 
%% Vorlage für kleinere Arbeiten wie Proseminar, Hausarbeit etc.
%% Für Bachelor/Masterarbeiten ist es besser, die Templates
%% https://github.com/ugroh/AGFA-Master zu nutzen 
%% Basis: KOMA-Script https://komascript.de/~mkohm/scrguide.pdf 
%% Stand: 2022/10/10
%% %%%%%%%%%%%%%%%%%%%%%%%%%%%%%%%%%%%%%%%%%%%%%%%%%%%%%%%%%%%%%%%% 
\documentclass[%				%  
	,ngerman					%  
%	,fontsize	= 12pt		% auskommentiert = 11pt		 	
	,parskip	= half-		% Absatz mit Leerzeile
	,BCOR		= 10mm		% Bindekorrektur fuer Klemmmappe etc
	,leqno					% Nummern der Formeln etc. Links
	,abstract	= true		% Zusammenfassung
	,headings	= small		% oder large oder normal
	,DIV		= calc		% Aufteilung der Seite überlassen wir der Anwendung
	]{scrartcl}   			% KOMA-Script als Basis; 

%% -- Die Vorlagen
%% --
%% %%%%%%%%%%%%%%%%%%%%%%%%%%%%%%%%%%%%%%%%%%%%%%%%%%%%%%%%%%
%% SEM-art.tex:
%% Layout, Font, Pakete etc.
%% Stand 2022/10/10
%% %%%%%%%%%%%%%%%%%%%%%%%%%%%%%%%%%%%%%%%%%%%%%%%%%%%%%%%%%%

\usepackage{onlyamsmath}			% Fehler bei Eingabe Mathematik
\usepackage[l2tabu,orthodox]{nag} 	% Fehler im LaTeX-Code

%% --------------------------
%% Falls kein UTF-8 eingestellt ist
%% Kann entfallen bei UTF8
%% --------------------------
\usepackage{selinput} 		% Automatische Wahl der "encoding"
	\SelectInputMappings{	% texdoc selinput	
		,adieresis={ä}		% 
       	,germandbls={ß}		%
            }
\usepackage[T1]{fontenc}   % texdoc fontenc

%% --------------------------
%% Wir schreiben deutschen Text und nutzen die deutschen Trennungsregeln
%% --------------------------
\usepackage[ngerman]{babel}
\usepackage[babel,german=guillemets]{csquotes} 		% \enquote{Text}
\babelprovide[hyphenrules=ngerman-x-latest]{ngerman}

%% --------------------------
%% Times New Roman
%% --------------------------
\usepackage{mathptmx}			
	\usepackage[scaled=.90]{helvet}
	\usepackage{courier}

%% --------------------------	
%% -- Alternative lmodern 
%% --------------------------	
%\usepackage{lmodern}
%\DeclareMathVersion{sans}
%	\SetSymbolFont{letters}{sans}{OML}{cmbr}{m}{it} % Math letters from Latin Modern Sans
%	\SetSymbolFont{operators}{sans}{OT1}{lmss}{m}{n} % Math operators
%	\SetSymbolFont{symbols}{sans}{OMS}{lmsy}{m}{n} % Math symbols
%	\SetMathAlphabet{\mathrm}{sans}{OT1}{lmr}{m}{n} % Large symbols
%	\SetMathAlphabet{\mathsf}{sans}{OT1}{lmss}{m}{n}
%	\SetMathAlphabet{\mathit}{sans}{OT1}{lmr}{m}{it}

%% --------------------------
%% Layout siehe KOMA-Script Buch für Details
%% --------------------------
\addtokomafont{title}{\let\huge\Large} 
\addtokomafont{author}{\large} 
\addtokomafont{date}{\normalsize}
% 
\renewcommand*{\thesubsection}{\arabic{subsection}.\kern1pt}
\RedeclareSectionCommand[font = \itshape]{subsection}
\setcounter{secnumdepth}{\subsubsectionnumdepth} 
%
\renewcommand*{\thesubsubsection}{\arabic{subsubsection}.\kern1pt} 
%
\RedeclareSectionCommand[%
  	,beforeskip		= .5\baselineskip 	% Etwas Abstand
	,afterskip		= -0.1em			% Spitzmarke, d.h. keine Absatz
	,font			= \normalfont	%	
	]{subsubsection} 

%% --------------------------
%% -- Kopf/Fußzeile
%% --------------------------

%% --------------------------
%% Sinnvolle Pakete
%% --------------------------

%% Bessere Aufzählungen 
\usepackage[shortlabels,inline]{enumitem} 	% texdoc enumitem

%% Flattersatz
\usepackage[%
	,newcommands
	,footnotes
	,raggedrightboxes
		]{ragged2e}	% Für Flattersatz

%% --------------------------
%% Mathematik
%% --------------------------
\usepackage{amsmath,amsthm,amssymb}

%% -- Mathematische Umgebungen	texdoc amsthm
\theoremstyle{plain}
%
\newtheorem{theorem}{Theorem}
\newtheorem{thm}[theorem]{Theorem}
\newtheorem{proposition}[theorem]{Satz}
\newtheorem{prop}[theorem]{Satz}
\newtheorem{satz}[theorem]{Satz}
\newtheorem{corollary}[theorem]{Korollar}
\newtheorem{cor}[theorem]{Korollar}
\newtheorem{korollar}[theorem]{Korollar}
\newtheorem{lemma}[theorem]{Lemma}
\newtheorem{lem}[theorem]{Lemma}
%
\theoremstyle{definition}
%
\newtheorem{definition}[theorem]{Definition}
\newtheorem{defn}[theorem]{Definition}
\newtheorem{example}[theorem]{Beispiel}
\newtheorem{beispiel}[theorem]{Beispiel}
%
\theoremstyle{remark}
%
\newtheorem*{note}{Anmerkung}
\newtheorem*{rem}{Anmerkung}
\newtheorem*{remark}{Anmerkung}

% Hier unnötig, da bereits über amsthm definiert
%\newtheorem{proof}{Beweis}
%\theoremsymbol{\ensuremath{\square}}















	% Layout etc. bitte nicht ändern
%% %%%%%%%%%%%%%%%%%%%%%%%%%%%%%%%%%%%%%%%%%%%%%%%%%%%%%%%%%%
%% Einige sinnvolle Definitionen
%% und nützliche Pakete
%% Eigene Ergäzungen in My-Def.tex
%% Stand: 2022/03/11
%% %%%%%%%%%%%%%%%%%%%%%%%%%%%%%%%%%%%%%%%%%%%%%%%%%%%%%%%%%%

%% -- Zahlen

\newcommand{\N}{\mathbb{N}}		% Natürliche Zahlen
\newcommand{\Z}{\mathbb{Z}}		% Ganze Zahlen
\newcommand{\Q}{\mathbb{Q}}		% Rationale Zahlen
\newcommand{\R}{\mathbb{R}}		% Reelle Zahlen
\newcommand{\C}{\mathbb{C}}		% Komplexe Zahlen
\newcommand{\K}{\mathbb{K}}		% Koerperzeichen

%% -- Ableitungen
%% -- auch als Beispieö für mathematische Ausdrücke
 
\newcommand*{\ds}{\mathop{}\!\mathrm{d}{s}}       % \ds = ds, 
\newcommand*{\dt}{\mathop{}\!\mathrm{d}{t}}       % \dt = dt, 
\newcommand*{\dx}{\mathop{}\!\mathrm{d}{x}}       % \dx = dx,
\newcommand*{\diff}[1]{\mathop{}\!\mathrm{d}{#1}}	% \diff{\mu} = d\mu, 

%% var-Symbole anstelle der Originalen, besser zu unterscheiden
%% --

\let\ORGvarepsilon=\varepsilon
\let\varepsilon=\epsilon
\let\epsilon=\ORGvarepsilon
%
\let\ORGvarrho=\varrho
\let\varrho=\rho
\let\rho=\ORGvarrho
%
\let\ORGvartheta=\vartheta
\let\vartheta=\theta
\let\theta=\ORGvartheta
%
\let\ORGvarphi=\varphi
\let\varphi=\phi
\let\phi=\ORGvarphi
%
\let\ORGvarleq=\leqslant
\let\leqslant=\leq
\let\leq=\ORGvarleq
%
\let\ORGvargeq=\geqslant
\let\geqslant=\geq
\let\geq=\ORGvargeq

%% -- Richtigen Abkürzungen; bitte auch in xspace-manual reinsehen
%% -- Bei Verwendung von \usepackage{hyperref} danch diesem Paket laden
%% --

\usepackage{xspace}
\newcommand{\zB}{z.\,B.\xspace}
\newcommand{\og}{o.\,g.\xspace}
\renewcommand{\dh}{d.\,h.\xspace}
\newcommand{\etc}{etc.\xspace}
\newcommand{\bzw}{bzw.\xspace}

	% Einige sinnvolle Definitioen
							% Siehe SEM-ReadMe.pdf für Erläuterungen

%% -- Eigene Definitionen
%% --
%% %%%%%%%%%%%%%%%%%%%%%%%%%%%%%%%%%%%%%%%%%%%%%%%%%%%%%%%
%% Die eigenen Definitionen
%% In Ergänzung zu AGFA-def.tex
%% Bitte vor eigen Definitionen diese Datei sich ansehen
%% Stand: 2022/03/10
%% %%%%%%%%%%%%%%%%%%%%%%%%%%%%%%%%%%%%%%%%%%%%%%%%%%%%%%%
%%	% Die eigenen Definitionen
							% Bitte die Definitionen in SEM-defn.tex beachten
%% --
\begin{document}

%% -- Die üblichen Angaben
%\pagestyle{headings}
\title{Titel der Ausarbeitung}
\subtitle{Für welche Vorlesung/Seminar etc.}
\author{Name}
\date{Datum} % Oder \today eingeben
%% --

\thispagestyle{empty}
\maketitle
%% --
% !TEX root = ../SEM-Master.tex
%% %%%%%%%%%%%%%%%%%%%%%%%%%%%%%%%%%%%%%%
%% Master.tex -> Inhalt meines Beitrags
%% in SEM-Master.tex File
%% Stand: 2022/10/10
%% %%%%%%%%%%%%%%%%%%%%%%%%%%%%%%%%%%%%%%
\thispagestyle{empty}
\begin{abstract}
\noindent
Hier steht eine kurze Zusammenfassung des Inhalts des Vortrags oder der Hausarbeit.
Dazu dieses \TeX{}-File kopieren und umbenennen.
Weitergehende Literatur ist im  Literatur\-verzeichnis aufgeführt, wobei ich das Buch von \textsc{Voss} \cite{voss-wiss} empfehlen kann.
Für einen Einstieg empfehle ich \emph{l2kurz.pdf}, das man auf 
%
\begin{center}
\texttt{http://mirror.ctan.org/info/lshort/german/} 
\end{center}
%
findet.
Weiteres in meinem Erläuterungen \emph{Sem-ReadMe.md}.
Ein Teil der hier angegeben Literatur und weiteres findet sich auf ILIAS \bzw in dem \og PDF.
\end{abstract}
%
\subsection{Erster Abschnitt}\label{sec:erster-abschnitt}
\subsubsection{}
Es genügt bei der Gliederung \verb|section{...}| und \verb|\subsection{...}| zu verwenden, gern auch in der $ ^{*} $-Variante.

Die weitere Untergliederung \verb|\subsubsection{}| wird nur zur \enquote{Nummerierung} genutzt (siehe das \TeX{}-File hierzu).
Bitte auch beachten: Ein Abschnitt sollte immer mindestens drei Unterabschnitte enthalten und diese dann auch mehrere Absätze. 

%
\subsubsection{} 
Ein weitere Unterabschnitt, indem dann endlich mit dem Vortrag eines Themas beginnt. 
Es sollte aber stets mit Hilfe des Befehls \verb|\cite{...}| auf die entsprechende Stelle des Textes verwiesen werden -- etwa \textsc{Oxtoby} \cite[Satz 7.8]{oxtoby}, dann findet man das Zitat leichter. 
%
\begin{center}
\verb|\textsc{Oxtoby} \cite[Satz 7.8]{oxtoby}|.
\end{center}
%
\subsubsection{} 
%
Und noch ein weiterer Unterabschnitt.
%
\subsection{Der Hauptsatz}
\subsubsection{}
Wir kommen nun zu unseren zentralen Satz der Mathematik.
%
\begin{theorem}\label{thm:hauptsatz}
%	
Ist $ f $ eine stetige reellwertige Funktion auf dem Intervall\/ $ \interval{0,1} $, so ist
%
\[
  	F( t ) = \int_{ 0 }^{ t } f(s) \ds
\]
%
differenzierbar auf diesem Intervall und $ F'(t) = f(t) $ für alle $ t \in  \interval{0,1} $.
\end{theorem}
%
\begin{proof}
Nun zum Beweis \ldots 
\end{proof}
%
\subsubsection{}
Für die Mathematik verwenden wir \textsc{Voss} \cite{voss-math} oder, da als PDF vorhanden \textsc{Grätzer} \cite{graetzer-ma}

\subsubsection{}
Für kleinere Ausarbeitungen ist die Verwendung der Möglichkeiten von \LaTeX{} für die Erstellung von Referenzen völlig ausreichend.
Für größere Arbeiten empfiehlt es sich, das Paket \texttt{biblatex} zu nutzen - siehe hierzu etwa meinen \LaTeX{}-Tipp Nr.5 und die dort angegeben Literatur.
Dieses Paket vereinfacht die Eingabe von Literaturzitaten erheblich.
 
\subsubsection{}
Weiteres findet sich in dem Eingangs erwähnten PDF-Dokument.

\subsection{Zusammenfassung und Ausblick}
Die Überschrift sagt alles \ldots


% -----------------------------------------------
\begin{thebibliography}{99}
%
\bibitem{graetzer-ma} George Grätzer,  
\emph{More Math into \LaTeX{}}, Springer (2007)
%
\bibitem{knuth:texbook} Donald E.Knuth,  
\emph{The \TeX{} Book}, Addison-Weseley (1986)
%
\bibitem{lambort} Leslie Lambort,
{\LaTeX}, \emph{Users Guide \& Reference Manual}, Addison-Weseley (1986) 
%
\bibitem{oxtoby} John C. Oxtoby,
\emph{Maß und Kategorie}, Springer (1971)
%
\bibitem{schlosser} Joachim Schlosser, 
\emph{Wissenschaftliche Arbeiten schreiben mit \LaTeX{} -- Leitfaden für Einsteiger} 6. Auflage, mitp (2016)
%
\bibitem{rrzn} Thomas Sturm, 
\emph{\LaTeX{} -- Einführung in das Textsatzsystem}, RRZN Hannover (2014)
%
\bibitem{voss-wiss} Herbert Voß, 
\emph{Die wissenschaftliche Arbeit mit \LaTeX{}} 2. Auflage, DANTE-Lehmanns media  (2021)
%
\bibitem{voss-math} Herbert Voß, 
\emph{Mathematiksatz mit \LaTeX{}}, 3. Auflage, DANTE-Lehmanns media (2018)
%%
\bibitem{voss-ref} Herbert Voß, 
\emph{\LaTeX{} Referenz}, 4. Auflage, DANTE-Lehmanns media  (2019)
%%
\end{thebibliography}
%%
	%% meine Ausführungen; durch eigene ersetzen, etwa 
%%% %%%%%%%%%%%%%%%%%%%%%%%%%%%%%%%%%%%%%%%%
%% MeinText.tex
%% Datei, in der man seinen Text schreiben kann
%% um diese dann per \input einzubinden
%% %%%%%%%%%%%%%%%%%%%%%%%%%%%%%%%%%%%%%%%%
\section{Erster Abschnitt}\label{sec:erster-abschnitt}
\subsection{Erster Unterabschnitt}






 %% 
%% --
\end{document}